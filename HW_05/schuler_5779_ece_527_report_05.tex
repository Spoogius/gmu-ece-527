\documentclass[11pt]{article}

    \usepackage[breakable]{tcolorbox}
    \usepackage{parskip} % Stop auto-indenting (to mimic markdown behaviour)
    

    % Basic figure setup, for now with no caption control since it's done
    % automatically by Pandoc (which extracts ![](path) syntax from Markdown).
    \usepackage{graphicx}
    % Keep aspect ratio if custom image width or height is specified
    \setkeys{Gin}{keepaspectratio}
    % Maintain compatibility with old templates. Remove in nbconvert 6.0
    \let\Oldincludegraphics\includegraphics
    % Ensure that by default, figures have no caption (until we provide a
    % proper Figure object with a Caption API and a way to capture that
    % in the conversion process - todo).
    \usepackage{caption}
    \DeclareCaptionFormat{nocaption}{}
    \captionsetup{format=nocaption,aboveskip=0pt,belowskip=0pt}

    \usepackage{float}
    \floatplacement{figure}{H} % forces figures to be placed at the correct location
    \usepackage{xcolor} % Allow colors to be defined
    \usepackage{enumerate} % Needed for markdown enumerations to work
    \usepackage{geometry} % Used to adjust the document margins
    \usepackage{amsmath} % Equations
    \usepackage{amssymb} % Equations
    \usepackage{textcomp} % defines textquotesingle
    % Hack from http://tex.stackexchange.com/a/47451/13684:
    \AtBeginDocument{%
        \def\PYZsq{\textquotesingle}% Upright quotes in Pygmentized code
    }
    \usepackage{upquote} % Upright quotes for verbatim code
    \usepackage{eurosym} % defines \euro

    \usepackage{iftex}
    \ifPDFTeX
        \usepackage[T1]{fontenc}
        \IfFileExists{alphabeta.sty}{
              \usepackage{alphabeta}
          }{
              \usepackage[mathletters]{ucs}
              \usepackage[utf8x]{inputenc}
          }
    \else
        \usepackage{fontspec}
        \usepackage{unicode-math}
    \fi

    \usepackage{fancyvrb} % verbatim replacement that allows latex
    \usepackage{grffile} % extends the file name processing of package graphics
                         % to support a larger range
    \makeatletter % fix for old versions of grffile with XeLaTeX
    \@ifpackagelater{grffile}{2019/11/01}
    {
      % Do nothing on new versions
    }
    {
      \def\Gread@@xetex#1{%
        \IfFileExists{"\Gin@base".bb}%
        {\Gread@eps{\Gin@base.bb}}%
        {\Gread@@xetex@aux#1}%
      }
    }
    \makeatother
    \usepackage[Export]{adjustbox} % Used to constrain images to a maximum size
    \adjustboxset{max size={0.9\linewidth}{0.9\paperheight}}

    % The hyperref package gives us a pdf with properly built
    % internal navigation ('pdf bookmarks' for the table of contents,
    % internal cross-reference links, web links for URLs, etc.)
    \usepackage{hyperref}
    % The default LaTeX title has an obnoxious amount of whitespace. By default,
    % titling removes some of it. It also provides customization options.
    \usepackage{titling}
    \usepackage{longtable} % longtable support required by pandoc >1.10
    \usepackage{booktabs}  % table support for pandoc > 1.12.2
    \usepackage{array}     % table support for pandoc >= 2.11.3
    \usepackage{calc}      % table minipage width calculation for pandoc >= 2.11.1
    \usepackage[inline]{enumitem} % IRkernel/repr support (it uses the enumerate* environment)
    \usepackage[normalem]{ulem} % ulem is needed to support strikethroughs (\sout)
                                % normalem makes italics be italics, not underlines
    \usepackage{soul}      % strikethrough (\st) support for pandoc >= 3.0.0
    \usepackage{mathrsfs}
    

    
    % Colors for the hyperref package
    \definecolor{urlcolor}{rgb}{0,.145,.698}
    \definecolor{linkcolor}{rgb}{.71,0.21,0.01}
    \definecolor{citecolor}{rgb}{.12,.54,.11}

    % ANSI colors
    \definecolor{ansi-black}{HTML}{3E424D}
    \definecolor{ansi-black-intense}{HTML}{282C36}
    \definecolor{ansi-red}{HTML}{E75C58}
    \definecolor{ansi-red-intense}{HTML}{B22B31}
    \definecolor{ansi-green}{HTML}{00A250}
    \definecolor{ansi-green-intense}{HTML}{007427}
    \definecolor{ansi-yellow}{HTML}{DDB62B}
    \definecolor{ansi-yellow-intense}{HTML}{B27D12}
    \definecolor{ansi-blue}{HTML}{208FFB}
    \definecolor{ansi-blue-intense}{HTML}{0065CA}
    \definecolor{ansi-magenta}{HTML}{D160C4}
    \definecolor{ansi-magenta-intense}{HTML}{A03196}
    \definecolor{ansi-cyan}{HTML}{60C6C8}
    \definecolor{ansi-cyan-intense}{HTML}{258F8F}
    \definecolor{ansi-white}{HTML}{C5C1B4}
    \definecolor{ansi-white-intense}{HTML}{A1A6B2}
    \definecolor{ansi-default-inverse-fg}{HTML}{FFFFFF}
    \definecolor{ansi-default-inverse-bg}{HTML}{000000}

    % common color for the border for error outputs.
    \definecolor{outerrorbackground}{HTML}{FFDFDF}

    % commands and environments needed by pandoc snippets
    % extracted from the output of `pandoc -s`
    \providecommand{\tightlist}{%
      \setlength{\itemsep}{0pt}\setlength{\parskip}{0pt}}
    \DefineVerbatimEnvironment{Highlighting}{Verbatim}{commandchars=\\\{\}}
    % Add ',fontsize=\small' for more characters per line
    \newenvironment{Shaded}{}{}
    \newcommand{\KeywordTok}[1]{\textcolor[rgb]{0.00,0.44,0.13}{\textbf{{#1}}}}
    \newcommand{\DataTypeTok}[1]{\textcolor[rgb]{0.56,0.13,0.00}{{#1}}}
    \newcommand{\DecValTok}[1]{\textcolor[rgb]{0.25,0.63,0.44}{{#1}}}
    \newcommand{\BaseNTok}[1]{\textcolor[rgb]{0.25,0.63,0.44}{{#1}}}
    \newcommand{\FloatTok}[1]{\textcolor[rgb]{0.25,0.63,0.44}{{#1}}}
    \newcommand{\CharTok}[1]{\textcolor[rgb]{0.25,0.44,0.63}{{#1}}}
    \newcommand{\StringTok}[1]{\textcolor[rgb]{0.25,0.44,0.63}{{#1}}}
    \newcommand{\CommentTok}[1]{\textcolor[rgb]{0.38,0.63,0.69}{\textit{{#1}}}}
    \newcommand{\OtherTok}[1]{\textcolor[rgb]{0.00,0.44,0.13}{{#1}}}
    \newcommand{\AlertTok}[1]{\textcolor[rgb]{1.00,0.00,0.00}{\textbf{{#1}}}}
    \newcommand{\FunctionTok}[1]{\textcolor[rgb]{0.02,0.16,0.49}{{#1}}}
    \newcommand{\RegionMarkerTok}[1]{{#1}}
    \newcommand{\ErrorTok}[1]{\textcolor[rgb]{1.00,0.00,0.00}{\textbf{{#1}}}}
    \newcommand{\NormalTok}[1]{{#1}}

    % Additional commands for more recent versions of Pandoc
    \newcommand{\ConstantTok}[1]{\textcolor[rgb]{0.53,0.00,0.00}{{#1}}}
    \newcommand{\SpecialCharTok}[1]{\textcolor[rgb]{0.25,0.44,0.63}{{#1}}}
    \newcommand{\VerbatimStringTok}[1]{\textcolor[rgb]{0.25,0.44,0.63}{{#1}}}
    \newcommand{\SpecialStringTok}[1]{\textcolor[rgb]{0.73,0.40,0.53}{{#1}}}
    \newcommand{\ImportTok}[1]{{#1}}
    \newcommand{\DocumentationTok}[1]{\textcolor[rgb]{0.73,0.13,0.13}{\textit{{#1}}}}
    \newcommand{\AnnotationTok}[1]{\textcolor[rgb]{0.38,0.63,0.69}{\textbf{\textit{{#1}}}}}
    \newcommand{\CommentVarTok}[1]{\textcolor[rgb]{0.38,0.63,0.69}{\textbf{\textit{{#1}}}}}
    \newcommand{\VariableTok}[1]{\textcolor[rgb]{0.10,0.09,0.49}{{#1}}}
    \newcommand{\ControlFlowTok}[1]{\textcolor[rgb]{0.00,0.44,0.13}{\textbf{{#1}}}}
    \newcommand{\OperatorTok}[1]{\textcolor[rgb]{0.40,0.40,0.40}{{#1}}}
    \newcommand{\BuiltInTok}[1]{{#1}}
    \newcommand{\ExtensionTok}[1]{{#1}}
    \newcommand{\PreprocessorTok}[1]{\textcolor[rgb]{0.74,0.48,0.00}{{#1}}}
    \newcommand{\AttributeTok}[1]{\textcolor[rgb]{0.49,0.56,0.16}{{#1}}}
    \newcommand{\InformationTok}[1]{\textcolor[rgb]{0.38,0.63,0.69}{\textbf{\textit{{#1}}}}}
    \newcommand{\WarningTok}[1]{\textcolor[rgb]{0.38,0.63,0.69}{\textbf{\textit{{#1}}}}}


    % Define a nice break command that doesn't care if a line doesn't already
    % exist.
    \def\br{\hspace*{\fill} \\* }
    % Math Jax compatibility definitions
    \def\gt{>}
    \def\lt{<}
    \let\Oldtex\TeX
    \let\Oldlatex\LaTeX
    \renewcommand{\TeX}{\textrm{\Oldtex}}
    \renewcommand{\LaTeX}{\textrm{\Oldlatex}}
    % Document parameters
    % Document title
    \title{schuler\_5779\_ece\_527\_report\_05}
    
    
    
    
    
    
    
% Pygments definitions
\makeatletter
\def\PY@reset{\let\PY@it=\relax \let\PY@bf=\relax%
    \let\PY@ul=\relax \let\PY@tc=\relax%
    \let\PY@bc=\relax \let\PY@ff=\relax}
\def\PY@tok#1{\csname PY@tok@#1\endcsname}
\def\PY@toks#1+{\ifx\relax#1\empty\else%
    \PY@tok{#1}\expandafter\PY@toks\fi}
\def\PY@do#1{\PY@bc{\PY@tc{\PY@ul{%
    \PY@it{\PY@bf{\PY@ff{#1}}}}}}}
\def\PY#1#2{\PY@reset\PY@toks#1+\relax+\PY@do{#2}}

\@namedef{PY@tok@w}{\def\PY@tc##1{\textcolor[rgb]{0.73,0.73,0.73}{##1}}}
\@namedef{PY@tok@c}{\let\PY@it=\textit\def\PY@tc##1{\textcolor[rgb]{0.24,0.48,0.48}{##1}}}
\@namedef{PY@tok@cp}{\def\PY@tc##1{\textcolor[rgb]{0.61,0.40,0.00}{##1}}}
\@namedef{PY@tok@k}{\let\PY@bf=\textbf\def\PY@tc##1{\textcolor[rgb]{0.00,0.50,0.00}{##1}}}
\@namedef{PY@tok@kp}{\def\PY@tc##1{\textcolor[rgb]{0.00,0.50,0.00}{##1}}}
\@namedef{PY@tok@kt}{\def\PY@tc##1{\textcolor[rgb]{0.69,0.00,0.25}{##1}}}
\@namedef{PY@tok@o}{\def\PY@tc##1{\textcolor[rgb]{0.40,0.40,0.40}{##1}}}
\@namedef{PY@tok@ow}{\let\PY@bf=\textbf\def\PY@tc##1{\textcolor[rgb]{0.67,0.13,1.00}{##1}}}
\@namedef{PY@tok@nb}{\def\PY@tc##1{\textcolor[rgb]{0.00,0.50,0.00}{##1}}}
\@namedef{PY@tok@nf}{\def\PY@tc##1{\textcolor[rgb]{0.00,0.00,1.00}{##1}}}
\@namedef{PY@tok@nc}{\let\PY@bf=\textbf\def\PY@tc##1{\textcolor[rgb]{0.00,0.00,1.00}{##1}}}
\@namedef{PY@tok@nn}{\let\PY@bf=\textbf\def\PY@tc##1{\textcolor[rgb]{0.00,0.00,1.00}{##1}}}
\@namedef{PY@tok@ne}{\let\PY@bf=\textbf\def\PY@tc##1{\textcolor[rgb]{0.80,0.25,0.22}{##1}}}
\@namedef{PY@tok@nv}{\def\PY@tc##1{\textcolor[rgb]{0.10,0.09,0.49}{##1}}}
\@namedef{PY@tok@no}{\def\PY@tc##1{\textcolor[rgb]{0.53,0.00,0.00}{##1}}}
\@namedef{PY@tok@nl}{\def\PY@tc##1{\textcolor[rgb]{0.46,0.46,0.00}{##1}}}
\@namedef{PY@tok@ni}{\let\PY@bf=\textbf\def\PY@tc##1{\textcolor[rgb]{0.44,0.44,0.44}{##1}}}
\@namedef{PY@tok@na}{\def\PY@tc##1{\textcolor[rgb]{0.41,0.47,0.13}{##1}}}
\@namedef{PY@tok@nt}{\let\PY@bf=\textbf\def\PY@tc##1{\textcolor[rgb]{0.00,0.50,0.00}{##1}}}
\@namedef{PY@tok@nd}{\def\PY@tc##1{\textcolor[rgb]{0.67,0.13,1.00}{##1}}}
\@namedef{PY@tok@s}{\def\PY@tc##1{\textcolor[rgb]{0.73,0.13,0.13}{##1}}}
\@namedef{PY@tok@sd}{\let\PY@it=\textit\def\PY@tc##1{\textcolor[rgb]{0.73,0.13,0.13}{##1}}}
\@namedef{PY@tok@si}{\let\PY@bf=\textbf\def\PY@tc##1{\textcolor[rgb]{0.64,0.35,0.47}{##1}}}
\@namedef{PY@tok@se}{\let\PY@bf=\textbf\def\PY@tc##1{\textcolor[rgb]{0.67,0.36,0.12}{##1}}}
\@namedef{PY@tok@sr}{\def\PY@tc##1{\textcolor[rgb]{0.64,0.35,0.47}{##1}}}
\@namedef{PY@tok@ss}{\def\PY@tc##1{\textcolor[rgb]{0.10,0.09,0.49}{##1}}}
\@namedef{PY@tok@sx}{\def\PY@tc##1{\textcolor[rgb]{0.00,0.50,0.00}{##1}}}
\@namedef{PY@tok@m}{\def\PY@tc##1{\textcolor[rgb]{0.40,0.40,0.40}{##1}}}
\@namedef{PY@tok@gh}{\let\PY@bf=\textbf\def\PY@tc##1{\textcolor[rgb]{0.00,0.00,0.50}{##1}}}
\@namedef{PY@tok@gu}{\let\PY@bf=\textbf\def\PY@tc##1{\textcolor[rgb]{0.50,0.00,0.50}{##1}}}
\@namedef{PY@tok@gd}{\def\PY@tc##1{\textcolor[rgb]{0.63,0.00,0.00}{##1}}}
\@namedef{PY@tok@gi}{\def\PY@tc##1{\textcolor[rgb]{0.00,0.52,0.00}{##1}}}
\@namedef{PY@tok@gr}{\def\PY@tc##1{\textcolor[rgb]{0.89,0.00,0.00}{##1}}}
\@namedef{PY@tok@ge}{\let\PY@it=\textit}
\@namedef{PY@tok@gs}{\let\PY@bf=\textbf}
\@namedef{PY@tok@ges}{\let\PY@bf=\textbf\let\PY@it=\textit}
\@namedef{PY@tok@gp}{\let\PY@bf=\textbf\def\PY@tc##1{\textcolor[rgb]{0.00,0.00,0.50}{##1}}}
\@namedef{PY@tok@go}{\def\PY@tc##1{\textcolor[rgb]{0.44,0.44,0.44}{##1}}}
\@namedef{PY@tok@gt}{\def\PY@tc##1{\textcolor[rgb]{0.00,0.27,0.87}{##1}}}
\@namedef{PY@tok@err}{\def\PY@bc##1{{\setlength{\fboxsep}{\string -\fboxrule}\fcolorbox[rgb]{1.00,0.00,0.00}{1,1,1}{\strut ##1}}}}
\@namedef{PY@tok@kc}{\let\PY@bf=\textbf\def\PY@tc##1{\textcolor[rgb]{0.00,0.50,0.00}{##1}}}
\@namedef{PY@tok@kd}{\let\PY@bf=\textbf\def\PY@tc##1{\textcolor[rgb]{0.00,0.50,0.00}{##1}}}
\@namedef{PY@tok@kn}{\let\PY@bf=\textbf\def\PY@tc##1{\textcolor[rgb]{0.00,0.50,0.00}{##1}}}
\@namedef{PY@tok@kr}{\let\PY@bf=\textbf\def\PY@tc##1{\textcolor[rgb]{0.00,0.50,0.00}{##1}}}
\@namedef{PY@tok@bp}{\def\PY@tc##1{\textcolor[rgb]{0.00,0.50,0.00}{##1}}}
\@namedef{PY@tok@fm}{\def\PY@tc##1{\textcolor[rgb]{0.00,0.00,1.00}{##1}}}
\@namedef{PY@tok@vc}{\def\PY@tc##1{\textcolor[rgb]{0.10,0.09,0.49}{##1}}}
\@namedef{PY@tok@vg}{\def\PY@tc##1{\textcolor[rgb]{0.10,0.09,0.49}{##1}}}
\@namedef{PY@tok@vi}{\def\PY@tc##1{\textcolor[rgb]{0.10,0.09,0.49}{##1}}}
\@namedef{PY@tok@vm}{\def\PY@tc##1{\textcolor[rgb]{0.10,0.09,0.49}{##1}}}
\@namedef{PY@tok@sa}{\def\PY@tc##1{\textcolor[rgb]{0.73,0.13,0.13}{##1}}}
\@namedef{PY@tok@sb}{\def\PY@tc##1{\textcolor[rgb]{0.73,0.13,0.13}{##1}}}
\@namedef{PY@tok@sc}{\def\PY@tc##1{\textcolor[rgb]{0.73,0.13,0.13}{##1}}}
\@namedef{PY@tok@dl}{\def\PY@tc##1{\textcolor[rgb]{0.73,0.13,0.13}{##1}}}
\@namedef{PY@tok@s2}{\def\PY@tc##1{\textcolor[rgb]{0.73,0.13,0.13}{##1}}}
\@namedef{PY@tok@sh}{\def\PY@tc##1{\textcolor[rgb]{0.73,0.13,0.13}{##1}}}
\@namedef{PY@tok@s1}{\def\PY@tc##1{\textcolor[rgb]{0.73,0.13,0.13}{##1}}}
\@namedef{PY@tok@mb}{\def\PY@tc##1{\textcolor[rgb]{0.40,0.40,0.40}{##1}}}
\@namedef{PY@tok@mf}{\def\PY@tc##1{\textcolor[rgb]{0.40,0.40,0.40}{##1}}}
\@namedef{PY@tok@mh}{\def\PY@tc##1{\textcolor[rgb]{0.40,0.40,0.40}{##1}}}
\@namedef{PY@tok@mi}{\def\PY@tc##1{\textcolor[rgb]{0.40,0.40,0.40}{##1}}}
\@namedef{PY@tok@il}{\def\PY@tc##1{\textcolor[rgb]{0.40,0.40,0.40}{##1}}}
\@namedef{PY@tok@mo}{\def\PY@tc##1{\textcolor[rgb]{0.40,0.40,0.40}{##1}}}
\@namedef{PY@tok@ch}{\let\PY@it=\textit\def\PY@tc##1{\textcolor[rgb]{0.24,0.48,0.48}{##1}}}
\@namedef{PY@tok@cm}{\let\PY@it=\textit\def\PY@tc##1{\textcolor[rgb]{0.24,0.48,0.48}{##1}}}
\@namedef{PY@tok@cpf}{\let\PY@it=\textit\def\PY@tc##1{\textcolor[rgb]{0.24,0.48,0.48}{##1}}}
\@namedef{PY@tok@c1}{\let\PY@it=\textit\def\PY@tc##1{\textcolor[rgb]{0.24,0.48,0.48}{##1}}}
\@namedef{PY@tok@cs}{\let\PY@it=\textit\def\PY@tc##1{\textcolor[rgb]{0.24,0.48,0.48}{##1}}}

\def\PYZbs{\char`\\}
\def\PYZus{\char`\_}
\def\PYZob{\char`\{}
\def\PYZcb{\char`\}}
\def\PYZca{\char`\^}
\def\PYZam{\char`\&}
\def\PYZlt{\char`\<}
\def\PYZgt{\char`\>}
\def\PYZsh{\char`\#}
\def\PYZpc{\char`\%}
\def\PYZdl{\char`\$}
\def\PYZhy{\char`\-}
\def\PYZsq{\char`\'}
\def\PYZdq{\char`\"}
\def\PYZti{\char`\~}
% for compatibility with earlier versions
\def\PYZat{@}
\def\PYZlb{[}
\def\PYZrb{]}
\makeatother


    % For linebreaks inside Verbatim environment from package fancyvrb.
    \makeatletter
        \newbox\Wrappedcontinuationbox
        \newbox\Wrappedvisiblespacebox
        \newcommand*\Wrappedvisiblespace {\textcolor{red}{\textvisiblespace}}
        \newcommand*\Wrappedcontinuationsymbol {\textcolor{red}{\llap{\tiny$\m@th\hookrightarrow$}}}
        \newcommand*\Wrappedcontinuationindent {3ex }
        \newcommand*\Wrappedafterbreak {\kern\Wrappedcontinuationindent\copy\Wrappedcontinuationbox}
        % Take advantage of the already applied Pygments mark-up to insert
        % potential linebreaks for TeX processing.
        %        {, <, #, %, $, ' and ": go to next line.
        %        _, }, ^, &, >, - and ~: stay at end of broken line.
        % Use of \textquotesingle for straight quote.
        \newcommand*\Wrappedbreaksatspecials {%
            \def\PYGZus{\discretionary{\char`\_}{\Wrappedafterbreak}{\char`\_}}%
            \def\PYGZob{\discretionary{}{\Wrappedafterbreak\char`\{}{\char`\{}}%
            \def\PYGZcb{\discretionary{\char`\}}{\Wrappedafterbreak}{\char`\}}}%
            \def\PYGZca{\discretionary{\char`\^}{\Wrappedafterbreak}{\char`\^}}%
            \def\PYGZam{\discretionary{\char`\&}{\Wrappedafterbreak}{\char`\&}}%
            \def\PYGZlt{\discretionary{}{\Wrappedafterbreak\char`\<}{\char`\<}}%
            \def\PYGZgt{\discretionary{\char`\>}{\Wrappedafterbreak}{\char`\>}}%
            \def\PYGZsh{\discretionary{}{\Wrappedafterbreak\char`\#}{\char`\#}}%
            \def\PYGZpc{\discretionary{}{\Wrappedafterbreak\char`\%}{\char`\%}}%
            \def\PYGZdl{\discretionary{}{\Wrappedafterbreak\char`\$}{\char`\$}}%
            \def\PYGZhy{\discretionary{\char`\-}{\Wrappedafterbreak}{\char`\-}}%
            \def\PYGZsq{\discretionary{}{\Wrappedafterbreak\textquotesingle}{\textquotesingle}}%
            \def\PYGZdq{\discretionary{}{\Wrappedafterbreak\char`\"}{\char`\"}}%
            \def\PYGZti{\discretionary{\char`\~}{\Wrappedafterbreak}{\char`\~}}%
        }
        % Some characters . , ; ? ! / are not pygmentized.
        % This macro makes them "active" and they will insert potential linebreaks
        \newcommand*\Wrappedbreaksatpunct {%
            \lccode`\~`\.\lowercase{\def~}{\discretionary{\hbox{\char`\.}}{\Wrappedafterbreak}{\hbox{\char`\.}}}%
            \lccode`\~`\,\lowercase{\def~}{\discretionary{\hbox{\char`\,}}{\Wrappedafterbreak}{\hbox{\char`\,}}}%
            \lccode`\~`\;\lowercase{\def~}{\discretionary{\hbox{\char`\;}}{\Wrappedafterbreak}{\hbox{\char`\;}}}%
            \lccode`\~`\:\lowercase{\def~}{\discretionary{\hbox{\char`\:}}{\Wrappedafterbreak}{\hbox{\char`\:}}}%
            \lccode`\~`\?\lowercase{\def~}{\discretionary{\hbox{\char`\?}}{\Wrappedafterbreak}{\hbox{\char`\?}}}%
            \lccode`\~`\!\lowercase{\def~}{\discretionary{\hbox{\char`\!}}{\Wrappedafterbreak}{\hbox{\char`\!}}}%
            \lccode`\~`\/\lowercase{\def~}{\discretionary{\hbox{\char`\/}}{\Wrappedafterbreak}{\hbox{\char`\/}}}%
            \catcode`\.\active
            \catcode`\,\active
            \catcode`\;\active
            \catcode`\:\active
            \catcode`\?\active
            \catcode`\!\active
            \catcode`\/\active
            \lccode`\~`\~
        }
    \makeatother

    \let\OriginalVerbatim=\Verbatim
    \makeatletter
    \renewcommand{\Verbatim}[1][1]{%
        %\parskip\z@skip
        \sbox\Wrappedcontinuationbox {\Wrappedcontinuationsymbol}%
        \sbox\Wrappedvisiblespacebox {\FV@SetupFont\Wrappedvisiblespace}%
        \def\FancyVerbFormatLine ##1{\hsize\linewidth
            \vtop{\raggedright\hyphenpenalty\z@\exhyphenpenalty\z@
                \doublehyphendemerits\z@\finalhyphendemerits\z@
                \strut ##1\strut}%
        }%
        % If the linebreak is at a space, the latter will be displayed as visible
        % space at end of first line, and a continuation symbol starts next line.
        % Stretch/shrink are however usually zero for typewriter font.
        \def\FV@Space {%
            \nobreak\hskip\z@ plus\fontdimen3\font minus\fontdimen4\font
            \discretionary{\copy\Wrappedvisiblespacebox}{\Wrappedafterbreak}
            {\kern\fontdimen2\font}%
        }%

        % Allow breaks at special characters using \PYG... macros.
        \Wrappedbreaksatspecials
        % Breaks at punctuation characters . , ; ? ! and / need catcode=\active
        \OriginalVerbatim[#1,codes*=\Wrappedbreaksatpunct]%
    }
    \makeatother

    % Exact colors from NB
    \definecolor{incolor}{HTML}{303F9F}
    \definecolor{outcolor}{HTML}{D84315}
    \definecolor{cellborder}{HTML}{CFCFCF}
    \definecolor{cellbackground}{HTML}{F7F7F7}

    % prompt
    \makeatletter
    \newcommand{\boxspacing}{\kern\kvtcb@left@rule\kern\kvtcb@boxsep}
    \makeatother
    \newcommand{\prompt}[4]{
        {\ttfamily\llap{{\color{#2}[#3]:\hspace{3pt}#4}}\vspace{-\baselineskip}}
    }
    

    
    % Prevent overflowing lines due to hard-to-break entities
    \sloppy
    % Setup hyperref package
    \hypersetup{
      breaklinks=true,  % so long urls are correctly broken across lines
      colorlinks=true,
      urlcolor=urlcolor,
      linkcolor=linkcolor,
      citecolor=citecolor,
      }
    % Slightly bigger margins than the latex defaults
    
    \geometry{verbose,tmargin=1in,bmargin=1in,lmargin=1in,rmargin=1in}
    
    

\begin{document}
    
    \maketitle
    
    

    
    \hypertarget{gmu-ece-527---computer-exercise-05---report}{%
\section{GMU ECE 527 - Computer Exercise \#05 -
Report}\label{gmu-ece-527---computer-exercise-05---report}}

\textbf{Stewart Schuler - G01395779}\\
\textbf{20241007}

    \hypertarget{exercise-5.1}{%
\subsection{Exercise 5.1}\label{exercise-5.1}}

\hypertarget{data-preparation}{%
\paragraph{Data Preparation}\label{data-preparation}}

The first task after importing the \emph{diabetes} dataset is to scale
the features to a roughly common scale. This can be done by two
different approaches. \emph{MinMax Normalization} which divides the
dataset but the largest value bounding the transformed features between
0 and 1. Or by \emph{Standard Normalization} which subtracks the mean
and divides by the variance. The result of apply the two different
normalization approaches the the dataset is shown in \emph{Figure 1}. It
can be seen that some of the features, namely 5 and 7, have a large
number of datapoints which fall very far outside the box plots quartile
range. In the case of \emph{Standard Normalization} these point being so
far away from the feature average means that post transformation the
features cover non-insignificant ranges. Because of this, for this
dataset \emph{MinMax Normalization} is the better approach to take. For
the remainder of this lab the data being applied to the regression
models will be \emph{MinMax} normalized.

\includegraphics{figures/5_3_box_plots.jpg}\\
\textbf{Figure 1.} Normalized Dataset Box Plots

Since we are doing regression it is helpful to maintain a separate
testset of data to be used to compare the regressors results to. Because
we want the test set to be completely isolated from any part of the
training procedure, when it comes to normalizing the test set we use the
normalization parameters extracted from the training set. In the case of
\emph{MinMax Normalization} which we are using, it is possible that a
feature value in the test set could be larger or smaller that those in
the training set. Which would lead to a value outside of the bounded 0
to 1 range. While not ideal, that must be the case to maintain test set
independence.

    \hypertarget{experiment-5.4}{%
\paragraph{Experiment 5.4}\label{experiment-5.4}}

Next we compute the correlation between all the features and with the
desired result \emph{Outcome}. This is presented in tabular for in
\emph{Table 1} and as a heatmap in \emph{Figure 2}.

\begin{longtable}[]{@{}
  >{\raggedright\arraybackslash}p{(\columnwidth - 18\tabcolsep) * \real{0.10}}
  >{\raggedright\arraybackslash}p{(\columnwidth - 18\tabcolsep) * \real{0.10}}
  >{\raggedright\arraybackslash}p{(\columnwidth - 18\tabcolsep) * \real{0.10}}
  >{\raggedright\arraybackslash}p{(\columnwidth - 18\tabcolsep) * \real{0.10}}
  >{\raggedright\arraybackslash}p{(\columnwidth - 18\tabcolsep) * \real{0.10}}
  >{\raggedright\arraybackslash}p{(\columnwidth - 18\tabcolsep) * \real{0.10}}
  >{\raggedright\arraybackslash}p{(\columnwidth - 18\tabcolsep) * \real{0.10}}
  >{\raggedright\arraybackslash}p{(\columnwidth - 18\tabcolsep) * \real{0.10}}
  >{\raggedright\arraybackslash}p{(\columnwidth - 18\tabcolsep) * \real{0.10}}
  >{\raggedright\arraybackslash}p{(\columnwidth - 18\tabcolsep) * \real{0.10}}@{}}
\toprule
\begin{minipage}[b]{\linewidth}\raggedright
\end{minipage} & \begin{minipage}[b]{\linewidth}\raggedright
Pregnancies
\end{minipage} & \begin{minipage}[b]{\linewidth}\raggedright
Glucose
\end{minipage} & \begin{minipage}[b]{\linewidth}\raggedright
BloodPressure
\end{minipage} & \begin{minipage}[b]{\linewidth}\raggedright
SkinThickness
\end{minipage} & \begin{minipage}[b]{\linewidth}\raggedright
Insulin
\end{minipage} & \begin{minipage}[b]{\linewidth}\raggedright
BMI
\end{minipage} & \begin{minipage}[b]{\linewidth}\raggedright
DiabetesPedigreeFunction
\end{minipage} & \begin{minipage}[b]{\linewidth}\raggedright
Age
\end{minipage} & \begin{minipage}[b]{\linewidth}\raggedright
Outcome
\end{minipage} \\
\midrule
\endhead
Pregnancies & 1.000000 & 0.129459 & 0.141282 & -0.081672 & -0.073535 &
0.017683 & -0.033523 & 0.544341 & 0.221898 \\
Glucose & 0.129459 & 1.000000 & 0.152590 & 0.057328 & 0.331357 &
0.221071 & 0.137337 & 0.263514 & 0.466581 \\
BloodPressure & 0.141282 & 0.152590 & 1.000000 & 0.207371 & 0.088933 &
0.281805 & 0.041265 & 0.239528 & 0.065068 \\
SkinThickness & -0.081672 & 0.057328 & 0.207371 & 1.000000 & 0.436783 &
0.392573 & 0.183928 & -0.113970 & 0.074752 \\
Insulin & -0.073535 & 0.331357 & 0.088933 & 0.436783 & 1.000000 &
0.197859 & 0.185071 & -0.042163 & 0.130548 \\
BMI & 0.017683 & 0.221071 & 0.281805 & 0.392573 & 0.197859 & 1.000000 &
0.140647 & 0.036242 & 0.292695 \\
DiabetesPedigreeFunction & -0.033523 & 0.137337 & 0.041265 & 0.183928 &
0.185071 & 0.140647 & 1.000000 & 0.033561 & 0.173844 \\
Age & 0.544341 & 0.263514 & 0.239528 & -0.113970 & -0.042163 & 0.036242
& 0.033561 & 1.000000 & 0.238356 \\
Outcome & 0.221898 & 0.466581 & 0.065068 & 0.074752 & 0.130548 &
0.292695 & 0.173844 & 0.238356 & 1.000000 \\
\bottomrule
\end{longtable}

\textbf{Table 1.} Feature Corrleation

\includegraphics{figures/5_4_heatmap.jpg}\\
\textbf{Figure 2.} Feature Correlation

From the above correlation we can see that \emph{Age/Pregnancies},
\emph{Insulin/SkinThickness} and \emph{BMI/SkinThickness} have the
highest correlation values between features. When correlated the
\emph{Outcome}, \emph{Glucose}, \emph{BMI}, and \emph{Age} have the
strongest correlation.

Next consider the pair plot in \emph{Figure 3} for the three features
identified as having the strongest correlation with \emph{Outcome}. It
can be seen that \emph{Glucose} has the strong separation between the
two distributions as we would expect given the strongest correlation
value. Secondly, the 2D plot of \emph{BMI} and \emph{Glucose} appears to
have the strongest separation between feature clusters. That would mean
those two features in combination would be a good candidate for a
reduced feature space model.

\includegraphics{figures/5_4_pair_plot.jpg}\\
\textbf{Figure 3.} Feature Pair Plot

    \hypertarget{experiment-5.5}{%
\paragraph{Experiment 5.5}\label{experiment-5.5}}

Next we apply the \emph{LogisticalRegressor} \emph{scikit-learn} class
to the dataset. We initially use all the available features, the result
of such a test are shown in \emph{Table 2}. For comparison, included is
the results using \emph{Standard Normalization} as well. The results
confirm our hunch that \emph{MinMax Normalization} would be better
suited for this dataset. However the difference is so small both would
likely be acceptable.

\begin{longtable}[]{@{}ll@{}}
\toprule
Normalization & Accuracy \\
\midrule
\endhead
MinMax & 78.57 \\
Standard & 77.38 \\
\bottomrule
\end{longtable}

\textbf{Table 2.} Logistical Regression Results - Full Features

Plotting the pseduo-probabilities computed by the classifier in
\emph{Figure 4} we can see a trend that the classifier makes predictions
with a much higher confidence when guessing an outcome of \(0\). This
can be analytically shown by computing the ratio of high confidence
predictions. Computed as follows for the two classes.

\[
\frac{\# Pred > 0.8}{\# Pred > 0.5} = 10.0\%
\] \[
\frac{\# Pred < 0.2}{\# Pred < 0.5} = 32.0\%
\]

Likewise \(73\%\) of all predictions fall in this \emph{low confidence}
region between 0.2 and 0.8.

\includegraphics{figures/5_5_distro.jpg}\\
\textbf{Figure 4.} Decision Probability

From these number it can be concluded that the classifier is more
confident when predicting an outcome of \(0\) rather than \(1\), in the
latter case the classifier is not very confident in it's answers. This
likely means that the decision boundary being learned by the classifier
well encompases the \(0\) class, but in doing so has a decent ammount of
\(1\) class samples contained in it.

    \hypertarget{experiment-5.6}{%
\paragraph{Experiment 5.6}\label{experiment-5.6}}

Next we consider the confusion matrix for the classifier. The results
it's present appear to line up with our intuition from the \emph{Figure
4}, in that the classifier is more confident about predicting class
\(0\) and less confident about class \(1\). The can be shown
analytically be comaring the inclass accuracies of the two classes. A
value of \(92.5\%\) for class \(0\) and only a \(53.3\%\) accuracy for
class \(1\).

To interpret the confusion matrix, the sum of each row indicates the
number of samples corresponding to the \emph{truth} value for each
class. That is there were \(108\) samples in the test dataset that have
an label of \(0\) and \(60\) samples that have a label of \(1\)\\
Likewise the sum of each column represent to predicted labels of the
test dataset. That is the classifier predicted \(128\) samples would be
class \(0\) and only \(40\) samples for class \(1\).

It can be clearly seen in \emph{Figure 5} that the confusion matrix is
not symmetic. Ideally we would want out classifier to by roughly
symmetic, a symmetic matrix means that the classifier performs
equivilantly on each of the classes. The imbalance displayed by our
matrix indicates it performs significantly better on one class (\(0\))
than the other.

\includegraphics{figures/5_6_cm.jpg}\\
\textbf{Figure 5.} Decision Probability

Next from the confusion matrix values we can compute some additional
metics. The \emph{Precision}, \emph{Recall}, and \emph{f1-score} found
in \emph{Table 3}.

\begin{longtable}[]{@{}ll@{}}
\toprule
Metric & Score \\
\midrule
\endhead
Precision & 0.80 \\
Recall & 0.53 \\
f1-score & 0.64 \\
\bottomrule
\end{longtable}

\textbf{Table 3.} Additional Metrics

From this additional discussion of results it can be seen how the raw
accuracy value can be misleading about the performance of a classifier.
The designed predictor performs well if the goal was only to correctly
identify input as class \(0\), which is the majority class in this
dataset leading to an inflated accuracy score. However we have shown
that the classifier does a very poor job at correctly predicting samples
belonging to class \(1\), in fact it only slightly out performs a 50-50
guess. Also taking into consideration the medical nature of this
dataset, the biggest risk when diagnosing a patient would be a
\emph{false negative}, which are abundant in this classifier. From that
we can conclude that this classifier is not very good.

Lastly we consider a further advanced metric the \emph{ROC} curve, shown
in Figure 6. In the \emph{ROC} curve our classifier with a \(\gamma\)
value of 0, lays at the point indicated by the green square. The ideal
classifier is a the top left corner of the plot, when \emph{True
Positive Rate} is 1, and \emph{False Positive Rate} is 0. With this is
mind it make sense to increase the \(\gamma\) value such that the
classifier moves closer to that point on the plot.

Also in the \emph{ROC} curve there is the red dashed line, this line
indicate a reference classifier for a classifier that chooses \(1\) for
every decision.

\includegraphics{figures/5_6_roc.jpg}\\
\textbf{Figure 6.} \emph{ROC} Curve

    \hypertarget{experiment-5.7}{%
\paragraph{Experiment 5.7}\label{experiment-5.7}}

To examine the learned inpact of each feature on the outcomes we can
plots the coefficients of the \(\bf{w}\) vector, as shown in
\emph{Figure 7}. From the plot we can see that \emph{Glucose} and
\emph{BMI} are the two most significant feature on the outcome. This is
inline with what we expect from our previous analysis. Interestingly
\emph{BloodPressure} has a negative impact on outcome, that is the
higher the blood pressure the more likely the result will be \(0\).

When comparing two features, \emph{BloodPressure} and \emph{Age}, to
determine which feature is more important there are two considerations
that must be taken. The first is the absolute value of the \(\bf{w}\)
vector contribution, from that we can be seen that \emph{Age} has a
\emph{marginally} higher impact on the result so we might be tempted to
say \emph{Age} is the more important feature. The second consideration
is that since \emph{BloodPressure} is the only negatively weighted
feature it's uniqueness give is more importance. For example, in a
normalized random sample with a true label of \(1\), the contribution of
\emph{Age} towards the final result being \(1\) would be dwarfed by the
other features with positive weights. Where as for a sample with a true
label of \(0\), where all the positivly weighted values contribute
minimally it is heavily dependent on the \emph{BloodPressure} feature to
drive the result down. Therefore, it can be concluded that
\emph{BloodPressure} is the more important of the two features.

\includegraphics{figures/5_7_feature_weights.jpg}\\
\textbf{Figure 7.} Feature Weights

    \hypertarget{experiment-5.8}{%
\paragraph{Experiment 5.8}\label{experiment-5.8}}

Next consider a specific input case, defined by the prenomalized values
in \emph{Table 4}. For this case the classifier predict a negative
diagnosis since the \(Pr\{y==1 | \bf{x}\} = 0.31\).

\begin{longtable}[]{@{}ll@{}}
\toprule
Feature & Value \\
\midrule
\endhead
Pregnancies & 0 \\
Glucose & 130 \\
BloodPressure & 125 \\
SkinThickness & 30 \\
Insulin & 100 \\
BMI & 32 \\
DiabetesPedigreeFunction & 1.1 \\
Age & 25 \\
Predicted Outcome & 0.31 \\
\bottomrule
\end{longtable}

\textbf{Table 4.} Specific Case Prediction

Finally we consider the predictor when trained and tested on a single
feature at a time. It can be seen in \emph{Figure 8} that those features
such as \emph{BMI} and \emph{Glucose} which had stronger weight values
(when trained on all features) have a steeper slope and cover the full
range of probabilities from 0 to 1. Where as those features which has
smaller weight values cover a smaller range with a flatter scope. And
Never really approach an outcome probabiltiy of 0 or 1 within a
reasonable feature rage.

When trained solely on the \emph{BloodPressure} feature, given the
result in \emph{Figure 7}, I would've expected the slope to be negative.
However that isn't that case, it does have the flattest slope of all the
features is still positive. That must mean for \emph{BloodPressure} to
negatively contribute to the outcome it must be coupled with some other
feature.

\includegraphics{figures/5_8_single_feature.jpg}\\
\textbf{Figure 8.} Feature Weights

    \hypertarget{extra-credit}{%
\paragraph{Extra Credit}\label{extra-credit}}

Consider next the impact of \emph{Age} on specific features. It is known
that \emph{Age} is positively correlated with an increased
\emph{Outcome} of \(1\). To visualize how much \emph{Age} modifies this
outcome we can plot the probability for a specific feature binned into
discrete age ranges. \emph{Figure 9} and \emph{10} show this for
\emph{Glucose} and \emph{BMI} respectively.

From \emph{Figure 10} it can be seen how much of an impact \emph{Age}
play in the prediction by comparing age ranges at the same \emph{BMI}
value. For example, for a \emph{BMI} of \(40\) there is nearly a \(0.4\)
difference in predicted outcome between the highest and lowest age
ranges. Where as in \emph{Figure 9} we see much smaller separation
between age ranges for the same value of \emph{Glucose}. Keeping that
observation in mind can be a useful dianostic tool.

\includegraphics{figures/ec_age_range_Glucose.jpg}\\
\textbf{Figure 9.} Glucose in Age Bins

\includegraphics{figures/ec_age_range_BMI.jpg}\\
\textbf{Figure 10.} BIM in Age Bins


    % Add a bibliography block to the postdoc
    
    
    
\end{document}
